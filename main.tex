\documentclass{article}
\usepackage[utf8]{inputenc}
\usepackage{lscape}
\usepackage{hyperref}
\usepackage{amsmath}
\hypersetup{
    colorlinks=true,
    linkcolor=blue,
    filecolor=magenta,      
    urlcolor=blue,
}
\usepackage{tikz}
\usepackage{indentfirst}
\usepackage[a4paper,margin={1.2in,1.5in},vmargin={1.2in,1.5in}]{geometry}
\geometry{paperwidth=210mm,paperheight=297mm,
textwidth=150mm,textheight=210mm,
top=23mm,bottom=23mm,
left=23mm,right=23mm}
\usepackage[colorlinks,linkcolor=blue,hyperindex]{hyperref}
%\usepackage[brazil]{babel}
\usepackage{graphicx,color}
\usepackage{multicol}
\usetikzlibrary{mindmap}
\pagestyle{empty}
\usepackage{fancyhdr}
\pagestyle{fancy}
\fancyhead[]{}}
\lhead{\thepage}
\fancyfoot[]{}}
\renewcommand{\headrulewidth}{0.1pt}
%=============================================================


%=============================================================
\begin{document}

\title{Upwork.com Template for a \LaTeX\, Document}

\author{\textit{by Rodrigo Hermont Ozon\footnote{Master Degree in Economic Development (2011) The Federal Parana University, see \href{http://lattes.cnpq.br/3532649625879285}{Currículo Lattes (only in portuguese)} or see my profile in LinkeDin \href{https://www.linkedin.com/in/rodrigohermontozon/}{here}}}}
\thispagestyle{empty}

\date{\today}

\begin{document}

\maketitle

\tableofcontents
\thispagestyle{empty}
\listoffigures
\listoftables
\newpage
%======================================================
\begin{center}
\includegraphics[width=4cm,height=4cm]{me}
\end{center}
%======================================================
\begin{abstract}
    Hello, and welcome to the sample \ LaTeX document demo page. I used Latex2HTML for conversion and made some adjustments to the html code then. In the following sections you will be able to check how I can easily edit your documents and turn them into a dynamic page, a pdf whether it is a book, a report, an article or a presentation.
\end{abstract}
%======================================================

\section{Using Excel2LaTeX for Gujarati Appendix C}

In this section I will replicate Appendix C of Damodar N. Gujarati's book Basic Econometrics, Fourth Edition (see references at end)~\cite{gujarati}, demonstrating the applicability of matrix calculation for OLS estimation using Excel spreadsheet.

(You can see the e-book  \href{https://himayatullah.weebly.com/uploads/5/3/4/0/53400977/gujarati\_book.pdf}{in this link})

\vspace{.5cm}
\subsection{C.10 SUMMARY OF THE MATRIX APPROACH:
AN ILLUSTRATIVE EXAMPLE}
\vspace{.5cm}

Consider the data given in Table C.4. These data pertain to per capita personal consumption expenditure (PPCE) and per capital personal disposable
income (PPDI) and time or the trend variable. By including the trend variable in the model, we are trying to find out the relationship of PPCE to PPDI
net of the trend variable (which may represent a host of other factors, such as technology, change in tastes, etc.)

For empirical purposes, therefore, the regression model is

$$
Y_{i}=\widehat{\beta}_{1}+\widehat{\beta}_{2}X_{2i}+\widehat{\beta}_{3}X_{3i}+\widehat{\mu}_{i}
$$


where $Y$ = per capita consumption expenditure, $X_{2} =$ per capita disposable income, and $X_{3} =$ time. The data required to run the regression are given in Table as follows:

% Table generated by Excel2LaTeX from sheet 'Tablec'
\begin{table}[htbp]
  \centering
  \caption{PER CAPITA PERSONAL CONSUMPTION EXPENDITURE (PPCE) AND PER CAPITA
PERSONAL DISPOSABLE INCOME (PPDI) IN THE UNITED STATES, 1956–1970,
IN 1958 DOLLARS}
    \begin{tabular}{ccc}
    \toprule
    \hline
    \textbf{PPCE, $Y$} & \textbf{ PPDI, $X_{2}$} & \textbf{Time, $X$} \\
    \midrule
    \hline
    1673  & 1839  & 1 ( = 1956) \\
    1688  & 1844  & 2 \\
    1666  & 1831  & 3 \\
    1735  & 1881  & 4 \\
    1749  & 1883  & 5 \\
    1756  & 1910  & 6 \\
    1815  & 1969  & 7 \\
    1867  & 2016  & 8 \\
    1948  & 2126  & 9 \\
    2048  & 2239  & 10 \\
    2128  & 2336  & 11 \\
    2165  & 2404  & 12 \\
    2257  & 2487  & 13 \\
    2316  & 2535  & 14 \\
    2324  & 2595  & 15 (= 1970) \\
    \bottomrule
    \hline
    \end{tabular}%
  \label{tab:addlabel}%
\end{table}%

\footnotesize \textit{Source: Economic Report of the President, January 1972, Table B-16.}

\normalsize
\newpage
In matrix notation, our problem may be shown as follows:

$$
\begin{bmatrix}
1673&\\
1688&\\
1666&\\
1735&\\
1749&\\
1756&\\
1815&\\
1867&\\
1948&\\
2048&\\
2128&\\
2165&\\
2257&\\
2316&\\
2324&\\
\end{bmatrix} = \begin{bmatrix}
1& 1839& 1&\\
1& 1844& 2&\\
1& 1831& 3&\\
1& 1881& 4&\\
1& 1883& 5&\\
1& 1910& 6&\\
1& 1969& 7&\\
1& 2016& 8&\\
1& 2126& 9&\\
1& 2239& 10&\\
1& 2336& 11&\\
1& 2404& 12&\\
1& 2487& 13&\\
1& 2535& 14&\\
1& 2595& 15&\\
\end{bmatrix}
\quad
\begin{bmatrix}
\widehat{\beta}_{1}&\\
\widehat{\beta}_{2}&\\
\widehat{\beta}_{3}&\\
\end{bmatrix}
+
\begin{bmatrix}
\widehat{u}_{1}&\\
\widehat{u}_{2}&\\
\widehat{u}_{3}&\\
\widehat{u}_{4}&\\
\widehat{u}_{5}&\\
\widehat{u}_{6}&\\
\widehat{u}_{7}&\\
\widehat{u}_{8}&\\
\widehat{u}_{9}&\\
\widehat{u}_{10}&\\
\widehat{u}_{11}&\\
\widehat{u}_{12}&\\
\widehat{u}_{13}&\\
\widehat{u}_{14}&\\
\widehat{u}_{15}&\\
\end{bmatrix}

\qquad\qquad\qquad\qquad\qquad\qquad \textbf{y}\quad\quad=\quad\quad\qquad \textbf{X}\quad\quad\qquad\qquad\widehat{\beta}\quad\quad+\qquad\widehat{\textbf{u}}
\\

\qquad\qquad\qquad\qquad\qquad\qquad 15\times 1\quad\qquad\quad15\times 3\quad\quad\qquad 3\times 1\qquad\quad15\times 1
$$

\vspace{.5cm}
From the preceding data we obtain the following quantities:

$$
\overline{Y}=1942.333 \quad \overline{X}_{2}=2126.333 \quad \overline{X}_{3}=8.0\\

\qquad \qquad\qquad\qquad \qquad\qquad\qquad\qquad  \sum (Y_{i}-\overline{Y})^{2}= 830,121.333\\
 
\qquad\qquad \qquad\qquad\qquad \sum(X_{2i}-\overline{X}_{2})^{2}= 1,103,111.333\qquad \sum(X_{3i}-\overline{X}_{3})^{2}=280.0 \\

\quad\qquad\qquad\qquad X^{'}X=\begin{bmatrix}
1&1&1&\ldots&1\\
X_2& 1 &X_2 &2 &X_2& 3 &\ldots &X_{2n}\\
X_3& 1 &X_3 &2& X_3& 3 &\ldots &X_{3n}\\
\end{bmatrix}
\begin{bmatrix}
1& X_{21}& X_{31}\\
1& X_{22}& X_{32}&\\
1& X_{23}& X_{33}&\\
\vdots&\vdots&\vdots\\
1& X_{2n}& X_{3n}
\end{bmatrix}\\
\vspace{.25cm}

\qquad\quad\quad\qquad\qquad\qquad=\begin{bmatrix}
n&\sum X_{2i}&\sum X_{3i}\\
\sum X_{2i}&\sum X_{2i}^{2}&\sum X_{2i}X_{3i}\\
\sum X_{3i}&\sum X_{2i}X_{3i}&\sum_{3i}^{2}
\end{bmatrix}\\
\vspace{.25cm}

\qquad\quad\quad\qquad\qquad\qquad=\begin{bmatrix}
15& 31,895& 120\\
31,895& 68,922.513& 272,144\\
120& 272,144& 1240
\end{bmatrix}\\
\vspace{.25cm}

\qquad\quad\quad\qquad\qquad X^{'} y=\begin{bmatrix}
29,135\\
62,905,821\\
247,934
\end{bmatrix}
$$

\vspace{.25cm}

Using the rules of matrix inversion given in Appendix B, one can see that

$$
(X^{'}X)^{-1}=\begin{bmatrix}
37.232491& -0.0225082& 1.336707\\
-0.0225082& 0.0000137& -0.0008319\\
1.336707& -0.0008319& 0.054034
\end{bmatrix}
$$
\vspace{.25cm}

Therefore,

$$
\widehat{\beta}=(X^{'}X)^{-1}X^{'}y=\begin{bmatrix}
300.28625\\
0.74198\\
8.04356
\end{bmatrix}
$$

\vspace{.25cm}
The residual sum of squares can now be computed as

$$
\sum \widehat{u}^{2}_{i}=\widehat{u}^{'}\widehat{u}\\

\quad\quad\quad\quad\qquad\qquad\qquad\qquad\qquad\qquad\qquad\qquad\qquad=y^{'}y-\beta^{'}X^{'}y\\

\quad\quad\quad\quad\qquad\qquad\qquad\qquad\qquad\qquad\qquad\qquad\qquad=57,420,003 - \begin{bmatrix}
300.28625& 0.74198& 8.04356
\end{bmatrix}
\begin{bmatrix}
29,135\\
62,905,821\\
247,934
\end{bmatrix}\\

\quad\quad\quad\quad\qquad\qquad\qquad\qquad\qquad\qquad\qquad\qquad\qquad= 1976.85574
$$

\vspace{.25cm}
whence we obtain

\vspace{.25cm}
$$
\widehat{\sigma}^{2}=\frac{\widehat{u}^{'}\widehat{u}}{12}=164.73797
$$

\vspace{.25cm}

The variance–covariance matrix for $\widehat{\beta}$ can therefore be shown as

$$
\mbox{var-cov}(\widehat{\beta})=\widehat{\sigma}^{2}(X^{'}X)^{-1}=\begin{bmatrix}
6133.650& −3.70794& 220.20634\\
− 3.70794& 0.00226& − 0.13705\\
220.20634& −0.13705& 8.90155
\end{bmatrix}
$$

The diagonal elements of this matrix give the variances of $\widehat{\beta}_{1},\,\widehat{\beta}_{2}$ and $\widehat{\beta}_{3}$ respectively, and their positive square roots give the corresponding standard errors.

From the previous data, it can be readily verified that

$$
\mbox{ESS:}\widehat{\beta}^{'}X^{'}y-n\overline{Y}^{2}= 828,144.47786 

$$
\mbox{TSS:}y^{'}y-n\overline{Y}^{2}= 830,121.333
$$

Therefore,

\begin{center}
\begin{align*}
    
R^{2}&=\displaystyle\frac{\widehat{\beta}^{'}X^{'}y-n\overline{Y}^{2}}{y^{'}y-n\overline{Y}^{2}}\\

\vspace{.25cm}
&=\displaystyle\frac{828,144.47786}{830,121.333}\\

\vspace{.25cm}
&= 0.99761
\end{align}
\end{center}

\vspace{-1cm}
Applying the \textbf{adjusted coefficient of determination} can be seen to be

$$
\overline{R}^{2}= 0.99722
$$

Collecting our results thus far, we have

\begin{align*}
\widehat{Y}_{i}&= 300.28625 + 0.74198X_{2i} + 8.04356X_{3i}\\
&\quad(78.31763)\quad(0.04753)\quad (2.98354)\\
t &= (3.83421)\quad (15.60956)\quad (2.69598)\\
R^2 &= 0.99761\quad \overline{R}^2 = 0.99722\quad df = 12
\end{align}

The interpretation of (C.10.14) is this: If both $X_2$ and $X_3$ are fixed at zero
value, the average value of per capita personal consumption expenditure is
estimated at about \$300. As usual, this mechanical interpretation of the intercept should be taken with a grain of salt. The partial regression coefficient of 0.74198 means that, holding all other variables constant, an increase in per capita income of, say, a dollar is accompanied by an increase in the mean per capita personal consumption expenditure of about 74 cents. In short, the marginal propensity to consume is estimated to be about 0.74, or 74 percent. Similarly, holding all other variables constant, the mean per capita personal consumption expenditure increased at the rate of about \$8 per year during the period of the study, 1956–1970. The $R^2$ value of 0.9976 shows that the two explanatory variables accounted for over 99 percent of the variation in per capita consumption expenditure in the United States over the period 1956–1970. Although  $\overline{R}^2$ dips slightly, it is still very high.



Turning to the statistical significance of the estimated coefficients, we see from (C.10.14) that each of the estimated coefficients is individually statistically significant at, say, the 5 percent level of significance: The ratios of the
estimated coefficients to their standard errors (that is, $t$ ratios) are 3.83421, 15.61077, and 2.69598, respectively. Using a two-tail $t$ test at the 5 percent level of significance, we see that the critical $t$ value for 12 $df$ is 2.179. Each of the computed $t$ values exceeds this critical value. Hence, individually we may reject the null hypothesis that the true population value of the relevant coefficient is zero.

As noted previously, we cannot apply the usual $t$ test to test the hypothesis that $\beta_{2}=\beta_{3}=0$ simultaneously because the $t-$ test procedure assumes that an independent sample is drawn every time the $t$ test is applied. If the same sample is used to test hypotheses about $\beta_{2}$ and $\beta_{3}$ simultaneously, it is likely that the estimators $\widehat{\beta}_{2}$ and $\widehat{\beta}_{3}$ are correlated, thus violating the assumption underlying the $t$-test procedure\footnote{See Sec. 8.5 for details.}. As a matter of fact, a look at the variance–covariance matrix of \textbf{$\widehat{\beta}$} given in (C.10.9) shows that the estimators $\widehat{\beta}_{2}$ and $\widehat{\beta}_{3}$ are negatively correlated (the covariance between the two is −0.13705).
Hence we cannot use the $t$ test to test the null hypothesis that $\beta_{2} = \beta_{3} = 0.$

But recall that a null hypothesis like $\beta_{2} = \beta_{3} = 0$, simultaneously, can be tested by the analysis of variance technique and the attendant $F$ test, which were introduced in Chapter 8. For our problem, the analysis of variance
table is Table C.5. Under the usual assumptions, we obtain

$$
F=\frac{414,072.3893}{164.73797}= 2513.52
$$

which is distributed as the $F$ distribution with 2 and 12 $df$. The computed $F$
value is obviously highly significant; we can reject the null hypothesis that
$\beta_{2} = \beta_{3} = 0$, that is, that per capita personal consumption expenditure is not linearly related to per capita disposable income and trend.
In Section C.9 we discussed the mechanics of forecasting, mean as well as individual. Assume that for 1971 the PPDI figure is \$2610 and we wish to forecast the PPCE corresponding to this figure. Then, the mean as well as individual forecast of PPCE for 1971 is the same and is given as

\begin{align*}
(PPCE_1971 | PPDI_1971, X_3 &= 16) = x^{'}_{1971}\widehat{\beta}\\
&\qquad\quad=\begin{bmatrix}
1 2610 16
\end{bmatrix}
\begin{bmatrix}
&300.28625\\
&0.74198\\
&8.04356\\
\end{bmatrix}\\
&\qquad\quad= 2365.55
\end{align*}

where use is made of (C.9.3).
The variances of $\widehat{Y}_{1971}$ and $\widehat{Y}_{1971}$, as we know from Section C.9, are different and are as follows:

\begin{align*}
    var(\widehat{Y}_{1971}|x^{'}_{1971})&=\widehat{\sigma}^{2}[x^{'}_{1971}(X^{'}X)^{-1}x_{1971}]\\
    & = 164.73797\begin{bmatrix}
    1 &2610& 16
    \end{bmatrix}(X^{'}X)^{-1}\begin{bmatrix}
    1\\
2610\\
16\\
    \end{bmatrix}
\end{align}

where $(X^{'}X)^{-1}$ is as shown in (C.10.5). Substituting this into (C.10.17), the reader should verify that
$$
var(\widehat{Y}_{1971}|x^{'}_{1971})=48.6426
$$

% Table generated by Excel2LaTeX from sheet 'anova'
\begin{table}[htbp]
  \centering
  \caption{THE ANOVA TABLE FOR THE DATA OF TABLE C.4}
    \begin{tabular}{llrr}
    \hline
    Source of variation  & SS    & \multicolumn{1}{l}{df} & \multicolumn{1}{l}{MSS} \\
    \hline
    Due to X2,X3  & 828,144.47786  & 2     & \multicolumn{1}{l}{414,072.3893} \\
    Due to residuals    & \underline{}1,976.85574} & \underline{12}    & \multicolumn{1}{l}{164.73797} \\
    Total  & 830,121.33360  & 14    &  \\
    \hline
    \end{tabular}%
  \label{tab:addlabel}%
\end{table}%

\newpage
and therefore

$$
se(\widehat{Y}_{1971}|x^{'}_{1971})=6.9744
$$

We leave it to the reader to verify, using (C.9.6), that

$$
var(Y_{1971}|x^{'}_{1971})= 14.6076
$$

\textit{Note:}$var(Y_{1971}|x^{'}_{1971})=E[Y_{1971}-\widehat{Y}_{1971}|x^{'}_{1971}]^2$. In Section C.5 we introduced the correlation matrix $R$. For our data, the
correlation matrix is as follows:

% Table generated by Excel2LaTeX from sheet 'matricorrel'
\begin{table}[htbp]
  \centering
    \begin{tabular}{cccc}
          & $Y$     & $X_{2}$    & \multicolumn{1}{l}{$X_{3}$} \\
          \hline
    $Y$\vline     & \multicolumn{1}{r}{1} & 0.9980 & \multicolumn{1}{l}{0.9743} \\
    $X_{2}$\vline    & 0.9980  & \multicolumn{1}{r}{1} & \multicolumn{1}{l}{0.9664} \\
    $X_{3}$\vline    & 0.9743  & 0.9664  & 1 \\
    \end{tabular}%
  \label{tab:addlabel}%
\end{table}%


Note that in (C.10.20) we have bordered the correlation matrix by the variables of the model so that we can readily identify which variables are involved in the computation of the correlation coefficient. Thus, the coefficient 0.9980 in the first row of matrix (C.10.20) tells us that it is the correlation coefficient between $Y$ and $X_{2}$ (that is, $r_{12}$). From the zero-order correlations given in the correlation matrix (C.10.20) one can easily derive the first-order correlation coefficients. (See exercise C.7.)
%=================================================================




%=================================================================
\newpage
\section{Using Excel for matricial approach, Gujarati Appendix C}








%=====================================================
\newpage
\begin{thebibliography}{99}
\bibitem{gujarati} Gujarati, D.,N. \textbf{Basic Econometrics}, fourth edition, McGraw-HiII/lrwin.

\end{thebibliography}

















%=================
\end{document}
%================